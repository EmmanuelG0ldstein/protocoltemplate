\documentclass[           %Befehl für die Dokumentendefinition
  titlepage=firstiscover, %ändert die Standardtitelseite
  10pt,a4paper]           %Schriftgröße und Papierformat 
  {scrartcl}              %Art des Dokuments, ein auf die deutsche Typograph                            ie zugeschnittenes Format

%ändert die Schriftart zu Latin Modern 
\usepackage{fontspec}

%Ermöglicht direkte Eingabe von Umlauten
\usepackage[utf8]{inputenc}

%Deutsche Sprachunterstützung
\usepackage[german]{babel}

%Europäische Erweiterung des Unicodes
\usepackage[T1]{fontenc}

%Mathematsiche Symbole etc.
\usepackage{amsmath}
\usepackage{amsfonts}
\usepackage{amssymb}

%Summenformeln
\usepackage{mhchem}

%ermöglicht das simple schreiben von Strukturformeln
\usepackage{chemfig}

%Einbetten von Grafiken
\usepackage{graphicx}

%Schönere, leserliche Tabellen, nicht so ein Massaker wie im PC Handbuch
\usepackage{booktabs}

%Zahlen und Einheiten nach Norm
\usepackage{siunitx}

%links innerhalb des  Dokuments
\usepackage[unicode]{hyperref}

%Veränderung der Fußnoten
\usepackage{bookmark}

%Veränderung der Überschrift von Tabellen
\usepackage{caption}

%Verlinkung bspw von Quellen möglich
\usepackage{url}

\usepackage{tikz}


\begin{document} %erst hiernach kann Text verfasst werden.
\title{Versuch XX}
\author{Name1 \and Name2 \and Gruppennummer XX}
\publishers{geschrieben von Name2} 
\date{%entweder von Hand bspw 1.1.1970 oder bspw per \today}

\maketitle              %erzeugt Titelseite
\thispagestyle{empty}   %Keine Nummeriung auf Titelseite
\newpage                %Seitenumbruch, damit Inhaltsverzeichnis etc nicht auf
                        %der titelseite erscheinen.
\tableofcontens         %Inhaltsverzeichnis
\listoffigure           %Abbildungsverzeichnis
\listoftables           %Tabellenverzeichnis

\newpage

\section{Einleitung}
\subsection{Theorie}
\subsubsection{Reaktionsgleichnug}  %LaTeX bietet 3 hierachische Abstufungen,
                                    %welche im Inhaltsverzeichnis dargestellt
                                    %werden
\section{Syntax}
Der \latex Syntax ist es extrem simpel gehalten, ein Befehl beginnt immer
mit einem " \ ", darauf folgt der Befehl, in [ ] folgen Optionen, welche
nicht obligatorisch sind, bspw die Schriftgröße für das Dokument 
(s. Zeile 1), dann folgt in { } das obligatorische Argument des Befehls.  
Jedes Pakte bringt seine eigenen Befehle (für mhchem bspw \ce) und 
eventuell seinen eigenen Syntax für das Argument mit sich, dafür gibt es 
die Paketdokumentation, dort sind auch Optionen etc aufgeführt. 
Vollständig sieht es dann so aus:
\befehl[Option1,Option2]{Argument}

Innerhalb einer Umgebung können neue Umgebungen geöffnet werden (Bspw
Schriftgröße innerhalb einer Tabelle), jedoch müssen Umgebungen immer wieder in
der umgekehrten Reihenfolge geschlossen werden:
\begin{table}
    \begin{tabular}{c}
        \begin{footnotesize}
            Text1 
            Text2 
        \end{footnotesize}
    \end{tabular}
\end{table}

Allgemein bekommen Tabellen eine ÜBERSCHRIFT (nicht wie im PC-Handbuch...),
welche mit \caption{ } erstellt wird. 
Mit \begin{table} erzeugt man eine sogenannte Gleitumgebung, welche von LaTeX
seperat betrachtet wird, dies ermöglicht übersichtlichere Dokumente.

Gleichungen sollten immer im mathmode geschrieben werden, dieser wird durch $ $
begrenzt.
So lassen sich bspw kürzere Reaktionen mit Strukturformel darstellen:
\begin{figure}

    \chemfig{*6(=-=(-[:30](-[:330](=[:270]\ce{NH})(-[:30]\ce{CH_3})))-=-)} +
    \ce{H_2}
    \schemestart \arrow{<=>} \schemestop
    \chemfig{*6(=-=(-[:30](-[:330](-[:270]\ce{NH})(-[:30]\ce{CH_3})))-=-)}·
    \ce{H_2}

\end{figure}


Aber auch kompliziertere Dinge sind darstellbar: 

\begin{figure}
  \begin{small}
    \chemfig{C(>[:50,,,,line width=1pt]C(-N(-[:110]C(-[:75]C(-\si{O^{-}}?[d])(=[:105]O)))(<:[:60]C(-[,,,,dash pattern=on 2pt off 2pt]C(=[:20]O)(-[:-30]\si{O^{-}}?[e])))(>:[7,,,,,red]\textcolor{red}{Ca}?[x]?[d,6,red]?[e,6,red])))(-[:-30,,,,line width=4pt]N?[x,7,red](-[6]C(-[7]C(-\si{O^{-}}?[x,7,red])(=[6]O)))(>[7,,,,line width=1pt]C(-[,,,,line width=2pt]C(-[:15]\si{O^{-}}?[x,7,red])(=[:-15]O))))}
  \end{small}
\end{figure}

Das mal als kurze Einführung in LaTeX, ich denke vorallem die Präambel mit den
zu verwendenden Paketen sollte verhindern, dass eure Dokumente nachher aussehen
wie das PC-Handbuch. 
\end{document}  %hiermit wird das Dokument beendet, danach sind nur noch
                %Kommentare möglich. Dieses Dokument lässt sich nicht
                %kompilieren, da die Beispiele keinen Befehlen entsprechen. 
                %Um dies zu erreichen einfach vor allen Klammern und 
                %Befehlen im Text % ergänzen, um diese auszukommentieren, dann beachtet
                %der Compiler diese nicht mehr.




